\section{Тестування та фактори якості програмного забезпечення}

\subsection{Основні поняття тестування}

Тестування програмного забезпечення представляє собою процес
дослідження програмного забезпечення (ПЗ) з метою встановлення
ступеню якості/готовності продукту для кінцевого замовника 
\cite{os-development,scrum-book,testing-black}.

Існує велика кількість підходів вирішення задач тестування.

Тестування може виконуватися на всіхстадіях життєвого циклу розробки,
на кожній з яких складають плани тестування, де описується кожний
елемент програми/системи, що повинен бути протестований, а саме:

\begin{itemize}
    \item дії, які потрібно виконати;
    \item тестові дані, які необхідно вводити;
    \item результат, що очікується в результаті тесту.
\end{itemize}

Частіше всього на тестування недостатньо виділяється часу,
щоб провести найбільш повне тестування. Тоді основна задача складається
в тому, щоб вибрати відповідний набір тестів, для проведення в
стислі строки. Для цього в тестуванні використовують
нарощувальний підхід. Тестування поділяється на стадії,
кількість яких пропорційна часу на тестування 
\cite{testing-black,test-automation-article}.

Задача тестувальників впевнитися, що програма чи продукт готові до випуску.
Тестувальник повинен звести до мінімуму кількість "неприємних сюрпризів",
які можуть виникнути після встановлення продукту у замовника.

З ISO 9126 \cite{iso-software-eng}, якість програмного продукту визначається як сукупна
характеристика програмного забезпечення, з урахуванням наступних факторів

\begin{itemize}
    \item основна функціональність;
    \item надійність;
    \item ефективність;
    \item практичність;
    \item можливість супроводжувати;
    \item практичність;
    \item мобільність;
    \item функціональність.
\end{itemize}

Основний список критеріїв і атрибутів було складено
у стандарті ISO9126 Міжнародної організації по стандартизації.
Склад і зміст документації, що супроводжує процес тестування,
визначається стандартом IEEE829-1998 "Standard for Software Test Documentation".

\subsection{Стадії тестування (Нарощувальний підхід при тестуванні)}

До стадій тестування відносять наступні \cite{testing-black,iso-software-eng}:

Стадія 1 - Вивчення - Ознайомлення з програмою/системою.

Стадія 2 -Базовий тест -Перевірка виконання основного тестового прикладу 
-Розробка і реалізація простого тестового прикладу,
що охвачує основні можливості функціоналу, які повинна виконувати система
перед тим як віддається на повне тестування.

Стадія 3 (необов’язкова, при необхідності) - Аналіз тенденцій
- Визначається, чи працює система/програма, як було заплановано,
коли ще неможливо попередньо оцінити реальні результати роботи.

Стадія 4 - Основна перевірка (інвентаризація) - визначення різних категорій даних,
створення та тестування тестів для кожного елементу категорії.

Стадія 5 - Комбінування вхідних даних - комбінування різних вхідних даних.

Стадія 6 - Граничне оцінювання - оцінювання поведінки програми при граничних значеннях даних.

Стадія 7 - Помилкові дані - Оцінювання реакції системи на введення неправильних даних.

Стадія 8 - Створення напруження - спроба вивести програму з ладу.

\subsection{Види тестування програмного забезпечення}

Класифікація видів тестування виконується за різними ознаками,
частіше всього виділяють:

За \emph{об’єктом тестування розрізняють}:

\begin{itemize}
    \item функціональне тестування (functional);
    \item тестування стабільності (stability / endurance / load);
    \item юзабіліті-тестування (usability);
    \item тестування локалізації (localization);
    \item тестування інтерфейсу користувача (UI);
    \item тестування продуктивності (performance);
    \item навантажувальне тестування (load);
    \item стрес-тестуванням (stress);
    \item тестування безпеки (security);
    \item тестування сумісності (compatibility testing).
\end{itemize}

За \emph{знанням систем} - чи має розробник доступ до коду програмного забезпечення:

\begin{itemize}
    \item тестування чорного ящика (black box) - немає доступу до коду,
      доступ тільки через інтерфейси, до яких має доступ як замовник, так й користувач;
    \item тестування сірого ящика (grey box) - є доступ до коду,
      але при безпосередньому виконанні тестів доступ до коду непотрібний;
    \item тестування білого ящика (white box testing) - є доступ до коду
      та виконувач може писати код, використовуючи розроблені бібліотеки;
\end{itemize}

За \emph{ступенем автоматизації}:

\begin{itemize}
    \item ручне тестування (manual);
    \item автоматизоване тестування (automated);
    \item напів-автоматизоване тестування (semiautomated testing).
\end{itemize}

За \emph{ступенем ізольованості компонентів}:

\begin{itemize}
    \item компонентне (модульне) тестування (component/unit);
    \item інтеграційне тестування (integration);
    \item системне тестування (system/end-to-endtesting).
\end{itemize}

За \emph{часом проведення}:

\begin{itemize}
    \item альфа-тестування (alpha testing);
    \item регресійне тестування (regression);
    \item тестування при прийомці (smoke testing);
    \item тестування нової функціональності(new feature testing);
    \item тестування при здачі (acceptance);
    \item бета тестування (beta testing).
\end{itemize}

За \emph{ознакою позитивності сценаріїв}:

\begin{itemize}
    \item позитивне тестування (positive testing);
    \item негативне тестування (negative testing).
\end{itemize}

За \emph{ступенем підготовленості до тестування}:

\begin{itemize}
    \item тестування по документації (formal testing);
    \item тестування ad hoc або інтуїтивне тестування (ad hoc testing).
\end{itemize}

При функціональному тестуванні виконується перевірка чи реалізовані
функціональні вимоги, тобто можливості програмного забезпечення,
в визначених у вхідній документації умовах, вирішувати завдання,
потрібні кінцевим користувачам. Функціональні вимоги визначають,
що саме робить продукт, які завдання вирішує.

Функціональні вимоги включають функціональну придатність; точність;
можливість до взаємодії; відповідність стандартам та правилам; безпеку.

При необхідності виконання тестування не функціональних параметрів
програми -створюються/описуються тести, необхідні для визначення
характеристик ПЗ, що можуть бути виміряні різними додатковими
елементами/величинами. До таких тестів відносять:

\begin{itemize}
    \item тестування продуктивності ПЗ - перевіряється працездатність:
      навантажувальне тестування, стресове тестування, тестування
      стабільності та надійності, швидкодії, об'ємне тестування,
      тестування на "відмову" та відновлення, конфігураційне тестування.
    \item тестування зручності використання виконується з метою
      визначення зручності використання програмного забезпечення
      для його подальшого застосування. Цей метод оцінки полягає
      у залученні користувачів як тестувальників-випробувачів і
      підсумовуванні отриманих від них висновків.
    \item тестування безпеки програм - перевірка конфіденційності
      даних для запобігання злому програми/продукту.
    \item тестування сумісності, де основною метою є
      перевірка коректної роботи продукту в певному середовищі.
      Середовище може включати в себе наступні елементи:
      різні браузери (Firefox, Opera, Chrome, Safari, Mozila, Internet Explorer);
      операційна система (Unix, Windows, MacOS); системне програмне
      забезпечення (веб-сервер, фаєрвол, антивірус);
      бази даних (Oracle, MS SQL, MySQL); периферія (принтери, CD/DVD-приводи, веб-камери).
\end{itemize}

Альфа-тестування – імітація реальної роботи програми/продукту
Найчастіше альфа-тестування проводиться на ранніх стадіях
розробки продуктів, але іноді може застосовуватися як внутрішнє
приймальне тестування. Виявлені помилки можуть бути
передані для додаткового дослідження у середовищі, подібному
тому, в якому буде використовуватисяпрограмне забезпечення, що тестується.

Бета-тестування - частіше всього, це приймальнетестування
вже на території замовника з метою виявлення помилок
на реальному середовищі, на проміжку між впровадженням
та введенням в промислову експлуатацію. Бета тестування
частіше всього проводиться на етапі дослідної експлуатації
на території замовника та іноді для того, щоб отримати
зворотній зв'язок про продукт від його майбутніх користувачів.

Часто стадія альфа-тестування характеризує функціональне
наповнення коду, а бета-тестування - стадію виправлення помилок.
При цьому, як правило, на кожному етапі розробки проміжні
результати роботи доступні кінцевим користувачам 
\cite{testing-black,iso-software-eng}.

\subsection{Фактори якості програмного забезпечення}

Фактор якості програмного забезпечення - це нефункціональна
вимога до програми, яка зазвичай не описується в договорі
з замовником, але тим не менше є бажаною вимогою, що підвищує
якість програмного продукту \cite{os-development,testing-black,test-automation-article}.

Розглянемо основні фактори якості:

\begin{itemize}
    \item зрозумілість - призначення програмного забезпечення 
      повинно бути зрозумілим із самої програми, та документації;
    \item повнота - всі необхідні частини програми
      повинні бути представлені і повністю реалізовані;
    \item стислість - відсутність зайвої інформації,
      та інформації, що дублюється;
    \item можливість перенесення - легкість адаптації
      програми до іншого середовища: іншої архітектури,
      платформи, операційної системи, або її версії;
    \item узгодженість - у всій програмі і в документації
      повинні використовуватись одні й ті самі узгодження, формати і визначення;
    \item можливість підтримувати супроводження - показник
      того, наскільки важко змінити програму для задоволення нових вимог.
      Ця вимога також показує, що програма повинна бути добре
      задокументована, не занадто заплутана, і мати резерв
      для росту при використанні ресурсів (пам’ять, процесор).
    \item тестованість - здатність програми здійснити перевірку
      приймальних характеристик: чи підтримується
      можливість виміру продуктивності?
    \item зручність використання - простота і зручність
      використання програмного продукту. Ця вимога
      в першу чергу відноситься до інтерфейсу користувача;
    \item надійність - відсутність відмов і збоїв у роботі
      програми, а також простота виправлення дефектів і помилок;
    \item ефективність - показник того, наскільки раціонально
      програма відноситься до ресурсів (пам’ять, процесор)
      при виконанні своїх задач.
\end{itemize}

Окрім технічного погляду на якістьпрограмного забезпечення,
існує і оцінка якості зі сторони користувача \cite{os-development,testing-black}.
Для цього аспекту якості іноді використовують термін "юзабіліті".
Доволі складно отримати оцінку "юзабіліті" для
заданого програмного продукту. Найбільш важливими питаннями,
що впливають на оцінку:

\begin{itemize}
    \item Чи являється інтерфейс користувача інтуїтивно зрозумілим?
    \item Наскільки просто виконувати прості, часті операції?
    \item Чи видає програма зрозумілі повідомлення про помилки?
    \item Чи завжди програма поводить себе так, як очікується?
    \item Чи є документація, і наскільки вона повна?
    \item Чи є інтерфейс користувача само-описуючим?
    \item Чи завжди затримки реакції програми є прийнятними?
\end{itemize}

