\anonsection{Вступ}

\emph{Актуальність.} При розробці програмного забезпечення однією з перешкод
є предметна галузь розробки. Фахівці з розробки програмного забезпечення
не можуть бути одночасно фахівцями предметних галузей різних задач.
Будь-яке програмне забезпечення має застосування в
тій чи іншій сфері діяльності або області інтересів.

Все частіше зустрічаються команди розробників, що займаються проектуванням
ПЗ по моделі предметної області. Предметно орієнтоване проектування це
підхід до розробки програмного забезпечення, що зосереджує розробку
на програмуванні моделі предметної області, що відображає бізнес­логіку 
\cite{ddd-evans, ddd-vernon, agile-book, scrum-guide, os-development, scrum-book}.

Для реалізації ПЗ з предметної області, яка невідома розробнику,
йому необхідно заповнити недолік, але не читаючи багато сторінкові
і малозрозумілі книги, наукові статті, оскільки це дасть розпливчасте уявлення.
Інструментом для подолання цих труднощів є модель,
яка будується з навмисно спрощених і строго відібраних знань.
Якщо модель дозволяє зосередитися на проблемі, то вона побудована правильно 
\cite{ddd-evans, ddd-vernon, agile-book, os-development}.

Актуальним є створення веб­додатку за використанням практик предметно орієнтованого проектування,
що дозволить розробникам краще розуміти бізнес модель проекту через використання
єдиної мови та термінів з експертами предметної області.

\emph{Метою роботи} є дослідження методик предметно орієнтованого проектування
на прикладі розробки веб-додатків.

\textit{Для досягнення мети необхідно розв'язати наступні задачі:}

\begin{enumerate}
  \item Провести аналіз існуючих практик проектування програмного забезпечення.
	\item Розробити багаторівневу архітектуру, для підвищення надійності,
		полегшення розробки нових модулів, та кращої тестувальності системи.
	\item Розробити модель домену вибраної предметної області.
	\item Розробити бізнес-правила додатку, що реалізують усі випадки
		використання системи.
	\item Розробити адаптери, які перетворюють дані з формату,
		найбільш зручного для випадків використання та сутностей,
		у формат, найбільш зручний для якоїсь зовнішньої служби.
	\item Реалізувати сервіси для взаємодії з зовнішніми службами.
\end{enumerate}

\emph{Об'єктом дослідження} є процеси проектування та розробки
програмного забезпечення з використанням практик 
предметно орієнтованого проектування на прикладі розробки веб-додатку.

\emph{Предметом дослідження} є методи та засоби проектування та
розробки програмного забезпечення з використанням предметно орієнтованого проектування.

\emph{Методи дослідження.} У роботі використовуються методи дослідження, а
саме аналіз, моделювання, класифікація, узагальнення, спостереження,
прогнозування та експерименту; методи передачі даних та методі
представлення результату.

\emph{Науково-технічний результат роботи} полягає в розробці удосконаленої
методики предметно орієнтованого програмного забезпечення на прикладі
розробки веб-додатків, що на відміну від існуючих дає можливість
покращити продуктивність проектування та розробки програмного 
забезпечення за рахунок покращення взаємодії команди розробників
з експертами предметної галузі. 

\emph{Практична цінність роботи} полягає в розробці бізнес-логіки
для покращення взаємодії команди розробників з експертами
предметної галузі для підвищення надійності, полегшення розробки
нових модулів та кращої тестувальності системи.

\emph{Апробація результатів роботи.} Результати даної роботи було
представлено на L науково-технічній конференції
факультету комп'ютерних систем і автоматики 
Вінницького національного технічного університету на
кафедрі автоматизації та інтелектуальних інформаційних технологій
та опубліковані у вигляді тез доповіді \cite{thesis} та впроваджено в ТОВ "ІКРОК" (додаток Є).
