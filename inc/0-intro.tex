\anonsection{Вступ}

\emph{Актуальність.} При розробці програмного забезпечення зустрічаєшся з різними труднощами.
Основна перешкода - це предметна область, в якій існує дана проблема.
Будь-яке програмне забезпечення має застосування в тій чи іншій сфері діяльності або області
інтересів. Наведемо приклад, найбільш пов'язаний з реальним світом:
онлайн покупка квитків на літак. Тобто предметна область - це така галузь знань,
в якій користувач використовує ПЗ.

Все частіше зустрічаються команди розробників, що займаються проектуванням ПЗ по моделі
предметної області. Предметно орієнтоване проектування це підхід до розробки програмного
забезпечення, який зосереджує розробку на програмуванні моделі предметної області, що відображає
бізнес-логіку \cite{ddd-article}.

Для реалізації ПЗ з предметної області, яка невідома розробнику, йому необхідно заповнити
недолік, але не читаючи багато сторінкові і малозрозумілі книги, наукові статті, оскільки
це дасть розпливчасте уявлення. Інструментом для подолання цих труднощів є модель, яка
будується з навмисно спрощених і строго відібраних знань. Якщо модель дозволяє зосередитися
на проблемі, то вона побудована правильно \cite{ddd-evans}.

Актуальним є створення веб-додатку за використанням практик предметно орієнтованого
проектування

\emph{Метою роботи} є дослідження методик предметно орієнтованого проектування
на прикладі розробки веб-додатків.

\textit{Для досягнення мети необхідно розв'язати наступні задачі:}

\begin{enumerate}
  \item Провести аналіз існуючих архітектурних рішень
	\item Розробити багаторівневу архітектуру, для підвищення надійності,
		полегшення розробки нових модулів, та кращої тестувальності системи.
	\item Розробити модель домену вибраної предметної області
	\item Розробити бізнес-правила додатку, що реалізують усі випадки
		використання системи.
	\item Розробити адаптери, які перетворюють дані з формату,
		найбільш зручного для випадків використання та сутностей,
		у формат, найбільш зручний для якоїсь зовнішньої служби.
\end{enumerate}

\emph{Практична цінність роботи} полягає в створенні

\emph{Апробація результатів та публікації.} За результатами даної роботи
опубліковано доповідь \cite{thesis} на науково-технічній конференції
Вінницького національного технічного університету,
факультету комп'ютерних систем і автоматики (м.Вінниця 2020).

\clearpage
