\section{Проектування структури та компонентів системи}

\subsection{Проектування багаторівневої архітектури}

Багаторівнева архітектура - одна з архітектурних парадигм розробки ПЗ,
при якій розбиття програми на самостійні складові частини відбувається
по реалізованої ними функціональності \cite{multitier-thesis}.

Характерні особливості багаторівневої архітектури:
\begin{itemize}
		\item Необхідна функціональність реалізується в одному рівні і не дублюється в інших.
		\item Кожен рівень повинен чітко реалізовувати ту функціональність,
			до області якої він відноситься, не поєднуючи код інших функціональних областей.
		\item Організація передачі даних між рівнями через компоненти доступу до даних,
			далі через бізнес-логіку, з передачею через контролюючі сервіси.
		\item Рівні слабко пов'язані між собою.
		\item Кожен рівень агрегує залежності і абстракції рівня,
			розташованого безпосередньо під ним.
		\item Фізично всі рівні можуть бути розгорнуті на одному комп'ютері або
			розподілені по різних комп'ютерах.
\end{itemize}

В логічно розділених на рівні архітектурах інформаційних систем найбільш
часто зустрічаються наступні рівні:

\begin{itemize}
		\item Рівень сутностей (даних) містить всі сутності, що використовуються в проектах програми.
		\item Рівень бізнес-логіки реалізує функціональні можливості програми.
		\item Сервісний рівень дозволяє використовувати додаток зовнішнім сервісам та рівню подання.
		\item Рівень користувальницького інтерфейсу (подання) надає ергономічний інтерфейс користувачу
			відповідно до функціоналу, описаному в технічному завданні.
		\item Рівень загальних компонентів містить всі бібліотеки і функціональні можливості,
			які можуть бути використані в будь-якому із зазначених вище рівнів.
\end{itemize}

Спроектована архітектура додатку зображена на рисунку \ref{app-arch}.
\addimghere{app-arch}{0.8}{Архітектура додатку}{app-arch}

\subsection{Моделювання бізнес-логіки}

Бізнес-логіка - це система зв'язків та залежностей елементів бізнес-даних
та правил обробки цих даних відповідно до особливостей ведення окремої діяльності (бізнес-правил),
яка встановлюється при розробці програмного забезпечення,
призначеного для автоматизації цієї діяльності. Бізнес логіка описує бізнес-правила реального світу,
які визначають способи створення, представлення та зміни даних.
Бізнес логіка контрастує з іншими частинами програми,
які мають відношення до низького рівня: управління базою даних,
відображення інтерфейсу користувача, інфраструктура і.т.д \cite{business-def}.

Перш ніж розпочати розробку бізнес-домену, ми повинні визначити
історії користувачів нашого додатку.

Історія користувача - це неформальне загальне пояснення функції
програмного забезпечення, написане з точки зору кінцевого користувача.
Його мета полягає в тому, щоб сформулювати, як функція програмного
забезпечення забезпечить цінність для клієнта \cite{user-story-article}.

Історії користувачів полегшують розуміння та спілкування між розробниками і можуть допомогти командам
документувати своє розуміння системи та її контексту.

Історії користувачів додатку:
\begin{itemize}
		\item Як учасник я хочу мати можливість створювати нову публікацію.
		\item Як учасник, я хочу мати можливість залишити новий коментар під публікацією.
		\item Як учасник, я хочу мати можливість відповідати на коментарі інших учасників.
		\item Як учасник, я хочу мати можливість бачити кількість переглядів у публікації.
		\item Як учасник я хочу мати можливість бачити загальну кількість вподобань певної публікації.
		\item Як учасник я хочу мати можливість бачити загальну кількість вподобань первного коментара.
		\item Як учасник я хочу мати можливість розміщувати лайки у публікації.
		\item Як учасник я хочу мати можливість розміщувати дизлайки у публікації.
		\item Як учасник я хочу мати можливість розміщувати лайки у коментарі.
		\item Як учасник я хочу мати можливість розміщувати дизлайки у коментарі.
\end{itemize}

Далі витягнемо іменники та дієслова із розповідей вище.
Шукаємо іменники, які стануть головними об'єктами, а не атрибутами.

Іменники:
\begin{itemize}
		\item Учасник
		\item Публікація
		\item Коментар
		\item Лайк
		\item Дизлайк
\end{itemize}

Дієслова:
\begin{itemize}
		\item Створити нову публікацію
		\item Залишити новий коментар
		\item Бачити загальну кількість лайків
		\item Бачити загальну кількість дизлайків
		\item Відповідати на коментар
		\item Розміщувати лайки
		\item Розміщувати дизлайки
		\item Бачити кількість переглядів публікації
\end{itemize}

Використовуючи вказані вище іменники та дієслова, ми можемо скласти схему (рисунок \ref{obj-inter-diagram})
\addimghere{obj-inter-diagram}{0.8}{Діаграма об'єктної взаємодії}{obj-inter-diagram}

Отримавши діаграму взаємодії об'єктів, ми можемо почати думати про діаграму відповідальності об'єкта.
Однією з найпоширеніших помилок є покладання відповідальності на об'єкт актора, тобто учасника.
Потрібно пам'ятати, що об'єкти повинні піклуватися про себе,
а також повинні бути закриті для безпосереднього спілкування.

Тож давайте слідувати вищезазначеному підходу та розподіляти обов'язки.
Діаграма відповідальності об'єкта зображена на рисунку \ref{obj-res-diagram}.
\addimghere{obj-res-diagram}{0.8}{Діаграма відповідальності об'єкта}{obj-res-diagram}

Тепер, коли ми створили діаграму взаємодії об'єктів та відповідальності,
ми можемо почати думати про UML діаграму класів. 
UML діаграма зображена на рисунку \ref{uml}.
\addimghere{uml-2}{0.8}{UML діаграма класів бізнес-логіки}{uml}

\subsection{Висновки}
У цьому розділі була розроблена архітектура додатку та створена модель бізнес-домену.
