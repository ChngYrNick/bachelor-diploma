\anonsection{Анотація}

У даній бакалаврській роботі проведене дослідження методів
предметно-орієнтованого проектування (domaindriven design, DDD)
та створено веб-додаток для кращого розуміння бізнес моделі
проекту через використання єдиної мови та термінів з
експертами предметної області. Розроблено багаторівневу
архітектуру, модель домену обраної предметної області,
бізнес-правила та реалізовані сервіси для взаємодії з зовнішними службами.

\vspace*{\baselineskip}

\anonsection{Abstract}
In this bachelor's thesis, a study of domain-driven
design (DDD) methods was conducted and a web application
was created to better understand the business model
of the project through the use of a ubiquitous language
and terms with domain experts. A layered
architecture, a domain model of the selected subject area,
business rules and implemented services for interaction
with external services have been developed.

\clearpage
