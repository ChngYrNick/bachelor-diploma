\anonsection{Висновки}

У даній бакалаврській роботі було проведене дослідження
методів предметно-орієнтованого проектуванняз метою аналізу
можливості покращення бізнес-моделей проекту через
використання єдиної мови та термінів з експертами предметної
області. Створено веб додаток для кращого розуміння бізнес
моделі проекту, розроблено багаторівневу архітектуру,
модель домейнуобраної предметної області, бізнес-правила
та реалізовані сервіси для взаємодії з зовнішними службами.

В першому розділі роботи було проаналізовано сучасний
стан проблеми та основні концепції предметно-орієнтованого програмування.

В другому розділі роботи було виконано проектування структури
та компонентів системи та моделювання бізнес-логіки предметної області.

В третьому розділі роботи було описано розробку компонентів системи,
реалізацію доменної події, розробку адаптерів перетворення даних,
розробку бізнес-логіки, взаємодію з зовнішніми службами.

В четвертому розділі описані процеси тестування
та визначені основні фактори якості розробленого програмного забезпечення.

Серед практиків, методи предметно-орієнтованого проектування
є загальновизнанимпідходом до побудовидодатків.
Застосування концепцій методів предметно-орієнтованого
програмування є складним завданням, оскільки йомуне
вистачає опису процесу розробки програмного забезпечення
та класифікаціїв рамках існуючих підходів до розробки програмного забезпечення.

В даній роботі удосконалено методику предметно-орієнтованого
програмного забезпечення на прикладі розробки веб-додатків,
що на відміну від існуючих дає можливість покращити продуктивність
проектування та розробки програмного забезпечення за рахунок
покращення взаємодії команди розробників з експертами предметної галузі.
