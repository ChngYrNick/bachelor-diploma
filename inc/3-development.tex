\section[РОЗРОБКА КОМПОНЕНТІВ СИСТЕМИ]{Розробка компонентів системи}

\subsection{Налаштування середовища розробки}
При розробці модулів додатку було використано мову програмування TypeScript.

TypeScript - це мова програмування, розроблена та підтримувана корпорацією Microsoft.
TypeScript є суворим синтаксичним набором JavaScript, який додає до мови необов'язкове статичне введення тексту.
TypeScript призначений для розробки великих додатків та перекомпіляції в JavaScript.
Оскільки TypeScript - це набір JavaScript, існуючі програми JavaScript також є дійсними програмами TypeScript
\cite{typescript-def}.

Типи які надає TypeScript дають спосіб описати форму об'єкта,
забезпечуючи кращу документацію та дозволяючи TypeScript перевірити, чи правильно працює код програми.

Для того щоб встановити TypeScript у директорії нашої програми, використаємо пакетний менеджер npm.

npm - це менеджер пакетів для середовища виконання Node.js мови програмування JavaScript.
Він складається з клієнта командного рядка, який також називається npm,
та онлайнової бази даних загальнодоступних та платних приватних пакетів, яка називається реєстр npm.
Доступ до реєстру здійснюється через клієнт, а доступні пакети можна
переглядати та шукати через веб-сайт npm.

Ініціалізуємо робочу директорії, та завантажуємо необхідні модулі використовуючи наступні команди:
\lstinputlisting[numbers=none,language=bash]{listings/init.example.sh}

Завантажуємо необхідні модулі з типами для TypeScript:
\lstinputlisting[numbers=none,language=bash]{listings/init2.example.sh}

Для того щоб скомпілювати TypeScript файли у JavaScript потрібно
скласти конфігураційний файл:
\lstinputlisting[numbers=none]{listings/tsconfig.json}

\subsection{Реалізація доменної події}

При розробці базового класу доменної події було використано клас EventEmitter модуля events.
Event emitter - це об'єкт / метод, який ініціює подію, як тільки відбувається якась певна дія,
щоб передати керування батьківській функції \cite{event-emitter-doc}.

EventEmitter ініціалізується таким чином:
\lstinputlisting[numbers=none,language=typescript]{listings/event-emitter.example.js}

Об'єкт класу EventEmitter має такі методи:

\begin{itemize}
    \item emit, активує певну подію;
    \item on, додає функцію зворотного виклику,
        яка буде виконуватися при активації події;
    \item once, додає одноразового слухача;
    \item removeListener, видаляє слухача події певної події;
    \item removeAllListeners, видалити всіх слухачів події.
\end{itemize}

Методи екземпляра класу EventEmitter були інкапсульовані 
у методах базового класу доменної події (DomainEvents).

Основні методи класу DomainEvents:

\begin{itemize}
    \item register, інкапсулює метод on об'єкта 
      класу EventEmitter і реєструє обробника певної події.
    \item dispatchAggregateEvents інкапсулює метод emit об'єкта
      класу EventEmitter та виконує обробника події для кожної
      зареєстрованої події певної сукупності.
    \item markAggregateForDispatch, реєструє сукупність в якій
      відбулася якась подія.
\end{itemize}

Реалізація базового класу доменної події наведено у додатку А.

\subsection{Розробка адаптерів перетворення даних}

При розробці адаптерів перетворення даних було використано 
архітектурний шаблон data mapper.

Data Mapper - це об'єкт який здійснює двонаправлену передачу даних між
постійним сховищем даних (часто реляційною базою даних)
та представленням даних у пам'яті (рівень домену).
Метою шаблону є збереження представлення в
пам'яті та постійного сховища даних незалежно один
від одного та від самого мапера даних.
Об'єкт складається з одного або декількох методів маперів,
що виконують передачу даних. Реалізації мапера відрізняється
за обсягом. Універсальні мапери оброблятимуть
багато різних типів сутностей домену,
спеціальні мапери - один або декілька.

Кожен адаптер перетворення даних має у складі два таких метода:

\begin{itemize}
    \item toDomain, перетворює дані у формат більш зручний для бізнес логіки;
    \item toPersistance, перетворює дані у формат більш зручний для певних зовнішніх служб.
\end{itemize}

Реалізація адаптерів перетворення даних наведена у додатку Д.

\subsection{Валідація даних}

Для валідації даних на відповідність бізнес правил було використано
шаблон проектування специфікація.

Специфікація - це шаблон проектування, за допомогою якого
уявлення правил бізнес логіки може бути перетворено у вигляді
ланцюжка об'єктів, пов'язаних операціями булевої логіки.

Цей шаблон виділяє такі специфікації (правила) в бізнес логіці,
які підходять для «зчеплення» з іншими. Об'єкт бізнес логіки
успадковує свою функціональність від абстрактного сукупного
класу AbstractSpecification, який містить всього один метод IsSatisfiedBy,
який повертає логічне значення. Після ініціалізування, об'єкт об'єднується
в ланцюжок з іншими об'єктами. В результаті, не втрачаючи гнучкості
в налаштуванні бізнес логіки, ми можемо з легкістю додавати нові правила.

Реалізація базового класу специфікації наведено у додатку А.

\subsection{Розробка бізнес-логіки}

При розробці сутностей було використано UML діаграму класів бізнес-логіки.
Для генерації ідентифікатора та підтримки унікальної ідентичності 
сутності було використано модуль uuid.

uuid - це модуль для генерації унікальних ідентифікаторів.

Приклад використання модуля uuid:
\lstinputlisting[numbers=none,language=typescript]{listings/uuid.example.js}

Методи об'єкта uuid були інкапсульовані в базовому класі UniqueID.

У кожній сутності, сукупності або об'єкта значення присутній статичний фабричний метод для створення такого об'єкта.
Також у фабричному методі об'єкта значення відбувається валідація вхідних даних на відповідність бізнес
правилам за допомогою шаблону специфікації.

\subsection{Взаємодія з зовнішніми службами}

Для реалізації доступу до бази даних було використано будівник запитів mysql2.

mysql2 - це драйвер Node.js для MySQL. mysql2 написаний на JavaScript, не вимагає компіляції.
mysql2 забезпечує виконання запитів до бази баних написаних мовою SQL.

Приклад базового використання mysql2:
\lstinputlisting[numbers=none,language=typescript]{listings/mysql2.example.js}

До кожної сутності системи було розроблено відповідний клас репозиторій
який інкапсулює методи модуля mysql2.

\subsection{Взаємодія через веб}

Для реалізації взаємодії з додатком через веб було використано веб-фреймворк Express.js.

Express.js - це мінімальний та гнучкий веб-фреймворк для розробки веб-додатків,
який забезпечує надійний набір функцій для веб і мобільних додатків.
Завдяки безлічі методів, утиліт HTTP та проміжного програмного
забезпечення, процес створення надійного API є швидким та простим.

Веб-фреймворки забезпечують такі ресурси, як HTML-сторінки, сценарії,
зображення тощо за різними маршрутами.
Наступна функція використовується для визначення маршрутів у Express.js:
\lstinputlisting[numbers=none,language=typescript]{listings/express.example.js}

Метод method можна застосувати до будь-якого з дієслів HTTP - get, set, put, delete.
Також існує альтернативний метод, який виконується незалежно від типу запиту.

Path - це маршрут, за яким буде виконуватися запит.

Handler - це функція зворотного виклику, яка виконується,
коли відповідний тип запиту знайдений на відповідному маршруті.

Приклад базового використання Express.js:
\lstinputlisting[numbers=none,language=typescript]{listings/express2.example.js}

Для обробки запитів були розроблені відповідні контролери. Контролери
у якості параметрів конструктора приймають екземпляри сервісів додатку.
Контролери після оброки запиту повертають результат роботи у якості відповіді на запит.
Реалізація контролерів наведено у додатку Г.
