\section{Розробка компонентів системи}

\subsection{Реалізація доменної події}

При розробці базового класу доменної події було використано клас EventEmitter модуля events.

Event emitter - це об'єкт / метод, який ініціює подію, як тільки відбувається якась певна дія,
щоб передати керування батьківській функції.

EventEmitter ініціалізується таким чином:
\lstinputlisting[numbers=none,language=typescript]{listings/event-emitter.example.js}

Об'єкт класа EventEmitter має такі методи:

\begin{itemize}
    \item emit, активує певну подію;
    \item on, додає функцію зворотного виклику,
        яка буде виконуватися при активації події;
    \item once, додає одноразового слухача;
    \item removeListener, видаляє слухача події певної події;
    \item removeAllListeners, видалити всіх слухачів події;
\end{itemize}

Реалізація базового класу доменної події зображено на додатку А.
