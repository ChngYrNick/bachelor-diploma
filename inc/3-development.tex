\section{Розробка компонентів системи}

\subsection{Налаштування середовища розробки}
При розробці модулів додатку було використано мову програмування TypeScript.

TypeScript - це мова програмування, розроблена та підтримувана корпорацією Microsoft.
TypeScript є суворим синтаксичним набором JavaScript, який додає до мови необов'язкове статичне введення тексту.
TypeScript призначений для розробки великих додатків та перекомпіляції в JavaScript.
Оскільки TypeScript - це набір JavaScript, існуючі програми JavaScript також є дійсними програмами TypeScript
\cite{typescript-def}.

Типи які надає TypeScript дають спосіб описати форму об’єкта,
забезпечуючи кращу документацію та дозволяючи TypeScript перевірити, чи правильно працює код програми.

Для того щоб встановити TypeScript у директорії нашої програми, використаємо пакетиний менеджер npm.

npm - це менеджер пакетів для середовища виконання Node.js мови програмування JavaScript.
Він складається з клієнта командного рядка, який також називається npm,
та онлайнової бази даних загальнодоступних та платних приватних пакетів, яка називається реєстр npm.
Доступ до реєстру здійснюється через клієнт, а доступні пакети можна
переглядати та шукати через веб-сайт npm.

Ініціалізуємо робочу директорії, та завантажуємо необхідні модулі використувуючи наступні команди:
\lstinputlisting[numbers=none,language=bash]{listings/init.example.sh}

Завантажуємо необхідні модулі з типами для TypeScript:
\lstinputlisting[numbers=none,language=bash]{listings/init2.example.sh}


\subsection{Реалізація доменної події}

При розробці базового класу доменної події було використано клас EventEmitter модуля events.

Event emitter - це об'єкт / метод, який ініціює подію, як тільки відбувається якась певна дія,
щоб передати керування батьківській функції \cite{event-emitter-doc}.

EventEmitter ініціалізується таким чином:
\lstinputlisting[numbers=none,language=typescript]{listings/event-emitter.example.js}

Об'єкт класа EventEmitter має такі методи:

\begin{itemize}
    \item emit, активує певну подію;
    \item on, додає функцію зворотного виклику,
        яка буде виконуватися при активації події;
    \item once, додає одноразового слухача;
    \item removeListener, видаляє слухача події певної події;
    \item removeAllListeners, видалити всіх слухачів події;
\end{itemize}

Реалізація базового класу доменної події наведено у додатку А.
